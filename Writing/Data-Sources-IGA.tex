\documentclass[12pt,notitlepage]{article}
\usepackage{bm,bbm,setspace,graphics,amsmath,rotating,amsfonts,amsxtra,amssymb,xcolor,hyperref,graphicx}

\hypersetup{breaklinks = true,
            bookmarks = true,
            pdfauthor = {Daniel Carpenter (Harvard University) and Devin Judge-Lord (University of Wisconsin) and Brian Libgober (Yale University) and Steven Rashin (New York University)},
             pdfkeywords  =  {Interest groups, administrative lobbying, bureaucratic politics, data
sources},  
            pdftitle = {Data and Methods for Analyzing Special Interest Influence in Rulemaking},
            colorlinks = true,
            citecolor = blue,
            urlcolor = blue,
            linkcolor = magenta,
            pdfborder = {0 0 0}}

\usepackage[small,compact]{titlesec}
\usepackage{geometry}
\usepackage{apptools}
\usepackage{lipsum,footmisc}
\usepackage{enumitem}
\usepackage[authoryear]{natbib}

\usepackage{textcomp}

\setcounter{MaxMatrixCols}{10}
\newcounter{theorem}
\newtheorem{theorem}[theorem]{Theorem}
\newtheorem{claim}[theorem]{Claim}
\newcounter{cnj}
\newtheorem{conjecture}[cnj]{Conjecture}
\newcounter{cor}
\newtheorem{corollary}[cor]{Corollary}
\newcounter{def}
\newtheorem{definition}[def]{Definition}
\newtheorem{example}[theorem]{Example}
\newcounter{lma}
\newtheorem{lemma}[lma]{Lemma}
\newcounter{prp}
\newtheorem{proposition}[prp]{Proposition}
\newcounter{rmk}
\newtheorem{remark}[rmk]{Remark}
\newcounter{obs}
\newtheorem{observation}[obs]{Observation}
\newenvironment{proof}[1][Proof]{\textbf{#1.} }{\ \rule{0.5em}{0.5em}}
%\topmargin=0 in \headheight=0in \headsep=0in \topskip=0in \textheight=9in \oddsidemargin=0in \evensidemargin=0in \textwidth=6.5in

\topmargin=0 in \headheight=0in \headsep=0in \topskip=0in \textheight=9in \oddsidemargin=0in \evensidemargin=0in \textwidth=6.5in



\usepackage{placeins} %%%% for floatbarrier
% \usepackage{titling} %%% for title and abstract on same page

%\usepackage{footmisc}
%\renewcommand{\footnotelayout}{\doublespacing} % set spacing in footnotes
%\newlength{\myfootnotesep}
%\setlength{\myfootnotesep}{\baselineskip}
%\addtolength{\myfootnotesep}{-\footnotesep}
%\setlength{\footnotesep}{\myfootnotesep} % set spacing between footnotes

\usepackage{booktabs}
\usepackage{caption}
\usepackage[normal]{threeparttable}

\usepackage{grffile}
\usepackage{xr}

%opening

%% Landscape mode for online appendix see https://stackoverflow.com/questions/25849814/rstudio-rmarkdown-both-portrait-and-landscape-layout-in-a-single-pdf
\usepackage{lscape}
\newcommand{\blandscape}{\begin{landscape}}
	\newcommand{\elandscape}{\end{landscape}}

% Keywords command
\providecommand{\keywords}[1]
{
	\small	
	\textbf{\textit{Keywords---}} #1
}

\usepackage{endnotes}


\let\footnote=\endnote


\usepackage{geometry}
\usepackage{setspace}
\usepackage[utf8]{inputenc}
\usepackage[english]{babel}
\setlength{\parindent}{4em}
\setlength{\parskip}{1em}
% \renewcommand{\baselinestretch}{2.0}
\usepackage{bookmark}

% title, author, date
  \title{Data and Methods for Analyzing Special Interest Influence in Rulemaking}

  \author{ % author, option footnote, optional affiliation
            Daniel Carpenter  \\ \emph{Harvard University} 
             \and 
           % author, option footnote, optional affiliation
            Devin Judge-Lord  \\ \emph{University of Wisconsin} 
             \and 
           % author, option footnote, optional affiliation
            Brian Libgober  \\ \emph{Yale University} 
             \and 
           % author, option footnote, optional affiliation
            Steven Rashin\footnote{Corresponding Author:
\href{mailto:Steven.Rashin@NYU.edu}{\nolinkurl{Steven.Rashin@NYU.edu}},
19 West 4th St, 2nd Floor, New York, NY 10012}  \\ \emph{New York University} 
            }


% auto-format date?
  \date{\today}




\begin{document}
	
%	\title{Data and Methods for Analyzing Special Interest Influence in Rulemaking}
% 	\author{true} % FIXME
	% \author{
	% 	Daniel Carpenter\thanks{Harvard University} \\
	% 	% 1737 Cambridge Street, 	 Room 405\\
	% 	% Cambridge, Massachusetts 02138 \\
	% 	\and
	% 	Devin Judge-Lord\thanks{University of Wisconsin} \\
	% 	% 1059 Bascom Mall, 110 North Hall \\
	% 	% Madison, WI 53706 \\
	% 	\and
	% 	Brian Libgober\thanks{Yale University}\\
	% 	% 115 Prospect Street, Room 113 \\
	% 	% New Haven, CT 06520 \\
	% 	\and
	% 	Steven Rashin\thanks{New York University, Corresponding Author: Steven.Rashin@NYU.edu} \\
	% 	% 19 West 4th St, 2nd Floor \\
	% 	% New York, NY 10012 \\
	% }


	
	\maketitle
	
	\singlespacing
	\begin{abstract}
	The United States government creates astonishingly complete records
relevant to policy creation in executive agencies. In this article, we
describe the major kinds of data that have proven useful to scholars
studying interest group behavior in bureaucratic politics, how to obtain
them, and also the major challenges that we as users have found in
working with these data. We discuss established databases such as
regulations.gov, which contains comments on regulations issued by
executive branch agencies, and new sources of data, such as ex-parte
meeting logs, which describe the interest groups and individual
lobbyists that bureaucrats are meeting face-to-face about rules, and
individually-identified personnel records of nearly all federal
employees since 1973. One of the challenges in working with the newer
datasets is that the whole databases are often not fully
machine-readable. We argue that a productive way forward is to invest in
making all the datasets machine-readable and to create consistent way to
link them to each other and to outside databases.
	\end{abstract} \hspace{10pt}
	
	%TC:ignore
	          \hfill \\ 
      \noindent \emph{Keywords}: Interest groups, administrative lobbying, bureaucratic politics, data
sources 
    	
	Submission ID: IGAD-D-19-00044
	\thispagestyle{empty}
	
	\newpage
	\clearpage
	\pagenumbering{arabic} 
	
	\doublespacing
	
	Data and Methods for Analyzing Special Interest Influence in Rulemaking
	
	\noindent Submission ID: IGAD-D-19-00044

If the U.S. federal government is unquestionably good at one thing, it
is pushing out paper. In theory, governmental records relevant to the
creation of policy in executive agencies have long been available to
researchers. Practically speaking, obtaining such data has been a
difficult and costly undertaking. A decade or more ago, researchers were
limited to acquiring data on just a few rules
\citep[e.g.,][]{Golden_JPART_1998} or using surveys
\citep[e.g.,][]{Furlong_JPART_2004}. Beginning in 1994, when the
government first released the \emph{Federal Register} online, scholars
now have access to astonishingly detailed data on the notice-and-comment
rulemaking process and, hence, regulatory policy, regulatory
policymakers, and interest group advocacy.

In this article, we describe the major kinds of data that have proven
useful to scholars studying interest group behavior in bureaucratic
politics, how to obtain them, and also the major challenges that we as
users have found in working with these data. Three examples of the data
sources that fit this description are machine-readable records of all
agency rules published since 1994, comments available on
regulations.gov, and metadata about rules contained in the Unified
Agenda. We also describe sources that have become more available in
recent years, such as ex-parte meeting logs, which describe the interest
groups and individual lobbyists that individual bureaucrats are meeting
face-to-face about rules, and individually-identified personnel records
of nearly all federal employees since 1973.

These data sources do not exhaust the kinds of records relevant to
researchers, although, at present, they reflect what is available. Yet
data accessibility increases all the time; for this reason, we also
highlight sources of data that we believe are already or may soon become
comprehensively available, perhaps after enterprising researchers submit
the necessary FOIA requests. Examples include agency press releases
\citep[see e.g.,][]{Libgober_JOP, Libgober_2018} and the Foreign Agents
Registration Act (FARA) reports \citep{Shepherd_APSR_2019}.\footnote{We
  have links to all the relevant data sources on our github page:
  \url{https://github.com/libgober/regdata/blob/master/README.md}.} To
orient potential students of agency policymaking to available sources of
data, we identify four units of analysis: (1) participants, (2)
policymakers, (3) policy texts, and (4) policy timing. We describe where
to find data on each in roughly the order they appear in the process of
developing a rule.

\hypertarget{background-rulemaking-and-the-administrative-procedures-act}{%
\subsection*{Background: Rulemaking and the Administrative Procedures
Act}\label{background-rulemaking-and-the-administrative-procedures-act}}
\addcontentsline{toc}{subsection}{Background: Rulemaking and the
Administrative Procedures Act}

For decades, the volume of legal requirements emerging from executive
agencies has dwarfed the law-making activity of Congress, the Supreme
Court, and the Presidency. Each year agencies publish four thousand or
more regulations. In doing so, agencies are generally required to follow
Section 553 of the Administrative Procedures Act (APA). This section
requires an agency to issue a Notice of Proposed Rulemaking (NPRM) in
the \emph{Federal Register} to notify potentially affected parties that
the regulatory environment might change. Following the NPRM, agencies
solicit comments on these regulations through regulations.gov (for
executive agencies) or the agency websites themselves (for independent
agencies). Agencies are required to consider the comments but are not
required to alter the rules based on them. The fact that the APA
requires agencies to assemble a comprehensive record of who sought
influence over any part of the policy-making process, and that such
records are indeed comprehensive, is highly unusual in the American
federal system, and makes the area a particularly exciting one for
interest group scholars.

\hypertarget{who-participates-in-the-rulemaking-process}{%
\section{Who participates in the rulemaking
process?}\label{who-participates-in-the-rulemaking-process}}

Notice and comment rulemaking creates unique feature in American
democracy; the right for anyone to participate in, perhaps even
influence, most policymaking processes at federal agencies.
Participation in rulemaking activities is highly skewed, with a few
rules, such as the Federal Communication Commission's rules on net
neutrality receive millions of comments while half of proposed rules
open for comment on regulations.gov receive no comments at all
\citep{Libgober_JOP}. Even the median rule designated as `economically
significant' -- rules projected to have an annual economic impact of
over \$100 million -- receives fewer than ten comments (Judge-Lord,
2019). Participants include businesses, public interest groups, trade
associations, unions, law firms, and academics
\citep{Cuellar_ALR_2005, Yackee_JOP_2006}.

Scholarly interest in patterns of participation in the policymaking
process relies on the notion that the participants in the process seek
to influence the development of the rule. Participation in this process
can begin even before a proposed rule is issued. For example,
\citet{You_JOP_2017} finds that half of all spending on lobbying
legislation occurs \emph{after} a bill becomes law. De
\citet{deFigureido_Kim_ICC_2004} show that meetings between agency
officials and firms spike before an agency issues a policy order.
\citet{Libgober_QJPS} shows that firms meeting with federal regulators
before a rule is issued may receive abnormally high stock market returns
upon its release, a finding consistent with the analysis of qualitative
researchers that commenters who participate early in the rulemaking
process can shape the content of the rule \citep{Naughton_JPAM_2009}. In
some cases public participation can even initiate a rulemaking process
via a petition for rulemaking.

To analyze patterns of participation, scholars use data from a variety
of sources. Sources for data on participation are agency rulemaking
dockets
\citep{Golden_JPART_1998, Yackee_JPART_2006, Young_BP_2017, Ban_BP_2019},
the \emph{Federal Register}
\citep[\citet{West_PAR_2004}]{Balla_APSR_1998}, and regulations.gov
\citep{Gordon_Rashin_JOP}. Though commenters are not generally required
to disclose their names and affiliations, many do. The best current data
sources for obtaining the names of the organizations that submit
comments on federal regulations are regulations.gov and the websites of
the independent agencies themselves.

\hypertarget{obtaining-and-working-with-data-on-comment-participants}{%
\subsection*{Obtaining and working with data on comment
participants}\label{obtaining-and-working-with-data-on-comment-participants}}
\addcontentsline{toc}{subsection}{Obtaining and working with data on
comment participants}

Scholars wishing to obtain data on comments and commenters from
executive agencies are required to use the website regulations.gov.
Actually obtaining the data requires overcoming a number of technical
and bureaucratic hurdles. First, bulk data from the website can only be
accessed via an Application Programming Interface (API) which requires
an API key.\footnote{An API is a set of procedures that allow a user to
  access data from a website in a structured way. Some websites limit
  API usage by requiring users to get an API key; regulations.gov is one
  of those websites.} Even after signing up for an API key,\footnote{By
  emailing
  \href{mailto:regulations@erulemakinghelpdesk.com}{\nolinkurl{regulations@erulemakinghelpdesk.com}}
  with `your name, email address, organization, and intended use of the
  API.' See \url{https://regulationsgov.github.io/developers/} for more
  details. The process to approve keys takes a few days, though some do
  not get approved at all.} we caution prospective users of the site
that these keys can be deactivated without warning or acknowledgment of
why they were deactivated. Second, obtaining an accurate count of
comments is not straightforward as agencies have different policies
regarding duplicated comments\footnote{We note that sometimes the
  difference between the total number of reported comments and the
  number of comments on regulations.gov is due to mass comment campaigns
  being grouped together.} and confidential business
information.\footnote{See e.g., \citet{Lubbers_2012} and
  \url{https://www.regulations.gov/userNotice}.} Third, obtaining the
data is time consuming.\footnote{Regulations.gov is subject to a rate
  limit of 1,000 queries per hour which is especially problematic for
  scholars seeking to analyze the full text of comments, many of which
  are included only as attachments. As of October 4, 2019
  Regulations.gov has 11,238,958 public submission documents
  representing over 70 million public comments and almost 1.5 million
  more rules and other documents. Downloading an individual document
  requires calling the API twice, once for the docket information and
  then a second time to download the linked file.}

Scholars studying participation outside of executive agencies can often
obtain these data from the websites of the agencies themselves. Unlike
regulations.gov, websites do not have an API and thus require scholars
to use bespoke web scrapers. We have examples of scrapers on
\url{https://github.com/libgober/regdata}. While these websites are not
rate limited or gated with API keys, we recommend exercising caution
with the rate at which a scholar solicits information from the servers,
as going too fast can result in a temporary ban.

In addition to the challenges of obtaining the data, scholars also must
choose how they want to preprocess the data before analyzing it. For
example, to answer questions about the types of organizations that
participate in notice and comment rulemaking, scholars first need to
accurately identify these organizations so that they can be assigned the
correct covariate data. This process, however, is not trivial as the
same firm can be identified in numerous ways. For example, Goldman Sachs
submitted comments to the Securities and Exchange Commission (SEC) and
Federal Reserve Board (FRB) as The Goldman Sachs Group, Inc, Goldman
Sachs Co.~LLC, Goldman Sachs Bank USA, Goldman, Sachs \& Co., Goldman
Sachs Execution \& Clearing, LP, among other identifiers. All of these
entities need to be accurately matched to Goldman Sachs, including the
Goldman Sachs Execution \& Clearing comment which is on Goldman Sachs
letterhead.\footnote{See
  \url{https://www.sec.gov/comments/s7-03-10/s70310-43.pdf}.} The
problem can be more acute with associations such as the American Bar
Association (ABA) sometimes commenting as itself and other times as
individual sections of the ABA. One way scholars have been able to get
around this problem is to use fuzzy matching. Fuzzy matching is a
process that compares two strings of text and gives a distance between
them. This procedure will assign a small distance between Goldman Sachs
and Goldman, Sachs \& Co and a larger distance between Goldman Sachs and
Merrill Lynch. The distances and optimal thresholds depend on the
algorithm used. Fuzzy matching, however, is not a panacea as the same
procedure will also show a small distance between the JP Morgan and
Morgan Stanley, which are two separate entities. It will also not reveal
whether an ABA letter is from the American Banker's Association or the
American Bar Association, and many trade-associations active on similar
rulemaking issues do unfortunately often share acronyms or have similar
looking names. We note, however, that in expectation, additional noise
generated from incorrect matching is likely to bias a scholar against
finding any relationships within the data.

\hypertarget{who-writes-rules}{%
\section{Who writes rules?}\label{who-writes-rules}}

In the last decade scholars have begun to study how the identities of
the policymakers and the networks in which they are embedded affect the
policymaking process. This recent scholarly attention is partially a
function of the release of new data sources such as the U.S. Office of
Personnel Management's (OPM) data on government employees
\citep[e.g.,][]{Bolton_AMP_2018},\footnote{Note that the dataset
  \citet{Bolton_AMP_2018} used is not public but there is a BuzzFeed
  version that is substantially similar that we discuss below.} Open
Secrets' lobbying \citep[e.g.,][]{Baumgartner_2009} and revolving door
databases \citep[e.g.,][]{Vidal_AER_2012, Bertrand_AER_2014}, machine
readable lobbying disclosure act reports
\citep[e.g.,][]{Boehmke_JPP_2013, You_JOP_2017}, meeting logs
\citep{Libgober_QJPS}, and datasets of corporate board membership such
as Boardex \citep[e.g.,][]{Shive_ROF_2016}. Scholars have also been able
to study the identities of policymakers through creative use of
longstanding data sources such as the \emph{Federal Register}
\citep[e.g.~][]{Carrigan_PAR_2019}.

Work on the networks between policymakers and private interests has
reshaped our understanding of the policymaking space. Through exploiting
meeting logs \citet{Libgober_QJPS} finds that comments and meetings with
policymakers are associated with abnormal returns in the billions.
\citet{Carrigan_PAR_2019} find that the number of job functions of the
bureaucrats who write the rules are associated with both decreases in
the time an agency takes to promulgate a rule and increases in the
probability that the rule will be overturned in court.
\citet{Vidal_AER_2012} find evidence that, among lobbyists, connections
to politicians are more valuable than issue expertise. In fact, when a
former Senate aide becomes a lobbyist the Senator's retirement results
in a substantial loss of income for that lobbyist.

\hypertarget{obtaining-and-working-with-personnel-and-lobbying-data}{%
\subsection*{Obtaining and Working with personnel and lobbying
data}\label{obtaining-and-working-with-personnel-and-lobbying-data}}
\addcontentsline{toc}{subsection}{Obtaining and Working with personnel
and lobbying data}

Scholars wishing to analyze the extent to which personnel influence the
development of the rulemaking process can obtain this data from the
BuzzFeed personnel data release. The BuzzFeed personnel data\footnote{Available
  at
  \url{https://archive.org/details/opm-federal-employment-data/page/n1}.
  Note that the complete dataset is massive (\$\textgreater\$30GB).}
contains information on federal employees such as their salaries, job
titles, and basic demographic data from 1973 through 2016.\footnote{Updated
  personnel files are available through 2018 from the OPM itself here:
  \url{https://www.fedscope.opm.gov/datadefn/index.asp}} These are the
most complete personnel records publicly available but they are subject
to several important limitations. First, not all agencies and
occupations are a part of this release.\footnote{The list includes at
  least 16 agencies and, within the covered agencies, law enforcement
  officers, nuclear engineers, and certain investigators (Buzzfeed,
  2017). See
  \url{https://www.buzzfeednews.com/article/jsvine/sharing-hundreds-of-millions-of-federal-payroll-records}}
Second, in some of the releases employees do not have a unique
identification number. This means that common names such as `John Smith'
match multiple employees. For example, just in the Veterans Health
Administration (VHA) there were 24 John Smiths in 2014. Adding a middle
initial still does not lead to unique identifiers as there are five John
Smiths at the VHA that have the same middle initial.

To obtain machine readable data on the identities of domestic lobbyists,
scholars use two databases from Open Secrets -- a non profit focused on
tracking spending on politics in the US -- on administrative and
Congressional lobbying. Downloading the lobbying data is straightforward
as it only requires an account to access the `bulk data' page. The
lobbying data comes from the required disclosures under the Lobbying
Disclosure Act of 1995 (LDA). Note that the reporting threshold varies
by type of firm (in-house have a higher minimum reporting threshold than
lobbying firms) and over time.\footnote{See
  \url{https://lobbyingdisclosure.house.gov/ldaguidance.pdf} for
  details.} The lobbying data covers 1999 through 2018 and is broken up
into seven tables that can easily be loaded into R or Python.\footnote{Note
  that the raw data can be downloaded from the Secretary of the Senate's
  Office of Public Records here:
  \url{https://www.senate.gov/legislative/Public_Disclosure/LDA_reports.htm}}
We note, however, that the data is not complete as some lobbyists do not
disclose required contacts and the data does not contain exact monetary
amounts.\footnote{See \url{https://www.gao.gov/assets/700/698103.pdf}}
We recommend Open Secret's database as it is easier to use than the raw
Senate data. Open Secrets also has a database on revolving door
employees that shows the career paths or federal government workers that
went to the private sector and are employed in some capacity where their
work depends on interacting with the federal government.\footnote{See
  \url{https://www.opensecrets.org/revolving/methodology.php} for
  details.} Unlike their lobbying database, this one must be scraped as
there is no option to download it using bulk data.\footnote{Note that
  they promote academic work using their database so they do not
  actively discourage scraping their database.}

Lobbyists advocating for foreign clients are required to disclose these
contacts under the Foreign Agent Registration Act. Foreign agents often
lobby bureaucratic agencies; for example, the state Israel retained law
firm Arnold \& Porter for advice on, among other issues, registering
securities with the SEC.\footnote{See
  \url{https://efile.fara.gov/docs/1750-Supplemental-Statement-20110729-13.pdf}}
The reports contain a multitude of data including the names foreign
entities, the firms representing them, the nature of the contact, and
the specific officials contacted. The data are astonishingly complete -
the website contains all 6264 FARA registrants and their foreign
contacts since July 3, 1942. The FARA data can be accessed online
through the Department of Justice.\footnote{\url{https://efile.fara.gov/ords/f?p=1381:1:13132679194789}:::::}
Unlike other data discussed in this essay, the FARA data is \emph{not}
all in machine readable form;\footnote{The metadata - e.g.~dates,
  registratrants, clients - are available in machine readable form
  through the API or bulk data downloads page.} converting from PDF to a
machine readable format (e.g.~csv) is neither easy nor error free. As a
result, scholars using the data only focus on a subset such as
\citet{You_2019} who focuses on lobbying activities by the governments
of Colombia, Panama, and South Korea on their free trade agreements from
2003 through 2012. In the absence of a completely digitized archive,
scholars can exploit the search functions to search by lobbing
registrant (e.g., lobbying firm Squire Patton Boggs) or foregin
principal (e.g., the government of Afghanistan). Similar to the
discussion of the diversity of names above, the same actor can
participate under different names. For example, the Embassy of the
Islamic Republic of Afghanistan and the Embassy of Afghanistan are
designated as separate entities under FARA.

Corporate executives, lawyers, and lobbyists often meet with agency
officials to discuss rules. Records of these meetings are often recorded
by agency personnel. Obtaining these data requires writing a web scraper
for each agency which holds the meetings data, since the data are held
in different places in each website. Since there are no uniform
standards for reporting meeting data, the data differ substantially from
agency to agency in both content and organization. The Federal Reserve,
for example, groups meetings by subject but not by rule,\footnote{\url{https://www.federalreserve.gov/regreform/communications-with-public.htm}}
the SEC groups meetings by rule, and the CFTC and FCC has all their
meetings in one place.\footnote{See
  \url{https://www.cftc.gov/LawRegulation/DoddFrankAct/ExternalMeetings?page=6}
  and
  \url{https://www.fcc.gov/proceedings-actions/ex-parte/archive-of-filings}}

\hypertarget{what-do-the-rules-say}{%
\section{What do the rules say?}\label{what-do-the-rules-say}}

All of the work discussed above relies on the notion that public
participation and the rulewrites influence the content of rules;
numerous studies have found that commenters influence rules \citep[see
e.g.,][]{Cuellar_ALR_2005, Yackee_JOP_2006, Naughton_JPAM_2009, Haeder_APSR_2015, Haeder_JPART_2018}
at agencies such as the Department of Labor \citep{Yackee_JOP_2006}, the
Department of Treasury \citep{Cuellar_ALR_2005}, the Environmental
Protection Agency \citep{Wagner_ALR_2011}, the Securities and Exchange
Commission \citep{Rashin_2019}, and numerous other agencies. This
finding is consistent across interviews
\citep[e.g.,][]{Furlong_JPART_1998}, qualitative analysis
\citep[e.g.,][]{Cuellar_ALR_2005}, and quantitative analysis
\citep[e.g.,][]{Yackee_JOP_2006}. Firms and groups representing
businesses' interests are influential in securing policy concessions
from the rulemaking process
\citep{Yackee_JPART_2006, Yackee_JOP_2006, Carpenter_2013}, particularly
when they are unopposed. Other scholars have found that agencies respond
to citizens \citep{Balla_PI_2019} and medical professionals
\citep{Balla_APSR_1998, Gordon_Rashin_JOP}.

\hypertarget{obtaining-and-working-with-data-on-rule-text}{%
\subsection*{Obtaining and Working with Data on Rule
Text}\label{obtaining-and-working-with-data-on-rule-text}}
\addcontentsline{toc}{subsection}{Obtaining and Working with Data on
Rule Text}

One of the most basic functions of the bureaucracy is to issue
regulations that set specific, enforceable requirements implementing a
law. The federal government publishes all rules in the \emph{Federal
Register}, accessible via FederalRegister.gov. This database is
comprehensive and has machine readable records of all regulations
published after 1994.\footnote{For issues from March 1939 through 1993,
  see \url{https://www.govinfo.gov/app/collection/fr/}, although note
  that they might require optical character recognition to make them in
  a readable format.} Scholars can access search for regulations by the
\emph{Federal Register} citation e.g.~47 FR 3431, \emph{Federal
Register} number, or, more usefully, through their API; unlike other
APIs mentioned in this essay, the \emph{Federal Register} API does not
have any rate limits.\footnote{However we still recommend small delays
  to prevent flooding the server with requests.}

While the raw text of rules is relatively straightforward to obtain,
there are several thorny theoretical and methodological issues scholars
must overcome.\footnote{We note that raw text of rules is not available
  from regulations.gov or the independent agencies themselves. As these
  rules are often in PDF form, converting them to .txt for analysis
  introduces errors into the text.} First, the standard path from notice
of proposed rulemaking (NPRM) to public comments to final rule is not
always straightforward. Some rules are withdrawn before a final rule.
Other agencies issue interim final rules subject to comments. For
studies that seek to compare the proposed and final rules, the most
problematic rules are the ones where one proposed rule gets broken up
into a few smaller final rules or the reverse, where small rules become
bundled into one final rule. These rules pose challenges for inference
as the processes that lead to amalgamation or separation are not well
understood. Second, scholars must decide whether, and to what extent,
they should preprocess the rulemaking data before feeding the data to a
text analysis algorithm.

\hypertarget{rule-metadata}{%
\section{Rule metadata}\label{rule-metadata}}

Rulemaking metadata, such as the time rules are released, allow scholars
to answer questions about factors that affect the rulemaking
environment. Scholars have exploited this data to study questions about
regulatory delay, agenda setting, and the financial impact of rules. The
often lengthy delays between statutory promulgation and the issuance of
regulations is the subject of recent empirical scholarly interest.
Scholars attribute various causes to regulatory delay including auditing
\citep{ACS_PSQ_2013}; outside contacts \citep{Balla_ALR_2011}; the
political climate \citep{Potter_JOP_2017, Thrower_PSQ_2018}; personnel
types \citep{Carrigan_PAR_2019}; ossification \citep{Yackee_GWLR_2012};
deadlines
\citep{Lavertu_JPART_2012, Carpenter_AJPS_2011, Bertelli_PAR_2019};
staffing levels \citep{Bolton_JLEO_2015}. These scholars relied on data
from OIRA
\citetext{\citealp[\citet{Balla_ALR_2011}]{ACS_PSQ_2013}; \citealp{Bolton_JLEO_2015}; \citealp{Carrigan_PAR_2019}},
the Unified Agenda
\citep{Bertelli_PAR_2019, Potter_JOP_2017, Lavertu_JPART_2012, Potter_2019},
the \emph{Federal Register} \citep{Yackee_GWLR_2012, Thrower_PSQ_2018},
and FDA's drug approval and postmarket experience
\citep{Carpenter_AJPS_2011}. Scholars have also exploited rule metadata
to show that publicly-traded banks that submit comments account for \$7
billion in excess returns \citep{Libgober_2018}.

\hypertarget{obtaining-and-working-with-the-unified-agenda-and-press-releases}{%
\subsection*{Obtaining and working with the Unified Agenda and Press
releases}\label{obtaining-and-working-with-the-unified-agenda-and-press-releases}}
\addcontentsline{toc}{subsection}{Obtaining and working with the Unified
Agenda and Press releases}

Obtaining data from the Unified Agenda is relatively straightforward as
all issues since 1995 are available online in machine readable
form.\footnote{The Unified Agenda from 1983 through 1994 is available in
  the \emph{Federal Register}.} The Unified Agenda contains all the
proposed regulations an agency plans to issue in the near future; making
it an extremely useful datasource for studying questions about agenda
setting and timing. There are, however, important limitations to this
data. First, agencies report their early stage rulemaking to the Unified
Agenda strategically \citep{Nou_SCLR_2016}. Second, agencies do not list
`failed' rules that did not become final rules in the Unified Agenda
\citep{Yackee_GWLR_2012}. Third, \citet{Coglianese_ALR_2016} note that
the Unified Agenda misses most of regulatory agency's work.

In addition to the disclosures mandated by law, agencies often issue
press releases to notify the public of important agency actions. Much
like the meetings data discussed above, the policies regarding their
storage and dissemination differ from agency to agency. The Federal
Reserve, for example, lists all press releases since 1996 on their
website, while the SEC only has them from 2012.\footnote{See
  \url{https://www.federalreserve.gov/newsevents/pressreleases.htm} and
  \url{https://www.sec.gov/news/pressreleases}} When working with the
press releases scholars often need data on the exact time documents were
made available to the general public \citep[see e.g.,][]{Libgober_JOP}.
Obtaining the press release metadata requires exploiting Really Simple
Syndication (RSS) feeds to extract the exact time a press release
becomes public.

\hypertarget{discussion-assembling-complete-databases}{%
\section{Discussion: Assembling Complete
Databases}\label{discussion-assembling-complete-databases}}

The United States government releases troves of data on rulemaking but
in forms that require substantial effort from scholars to be useful for
research. Scholars working on bureaucratic politics face two primary
data challenges going forward: assembling complete, machine-readable
datasets of agency rulemaking activity and linking these datasets to one
another. There are four fruitful data projects which could increase
researcher efficiency by preventing duplicated efforts to download and
clean data. First, assembling a complete database of comment activity
throughout the Federal government in one location that can be downloaded
in bulk, including comment text and metadata. Second, creating a
comprehensive database of revolving door rulemakers from the OPM
personnel records, LDA disclosure forms, and FARA data. This database
would be greatly helped by the addition of data from LinkedIn but that
data is not currently public. Third, creating a database of all meeting
activities throughout the federal government. Finally, creating unique
identifiers for each commenter that allow researchers to link commenting
behavior to firm characteristics and other datasets. These projects,
once complete, will allow scholars to pursue novel research on political
participation, influence, and public management.

\hypertarget{conflicts-of-interest}{%
\section{Conflicts of Interest}\label{conflicts-of-interest}}

On behalf of all authors, the corresponding author states that there is
no conflict of interest.
	
	\newpage
	
	\theendnotes
		
	\begingroup
	\parindent 0pt
	\parskip 2ex
	\def\enotesize{\normalsize}
	\pdfbookmark{Endnotes}{Endnotes}
	\endgroup
	
	\newpage
		\singlespacing
	\bibliographystyle{apsr2}
	\bibliography{data_ig.bib}
\end{document}